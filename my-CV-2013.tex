%% start of file `my-CV-2013.tex'.
%% Copyright 2006-2013 Xavier Danaux (xdanaux@gmail.com).
%
% This work may be distributed and/or modified under the
% conditions of the LaTeX Project Public License version 1.3c,
% available at http://www.latex-project.org/lppl/.

% possible options include font size ('10pt', '11pt' and '12pt'),
% paper size ('a4paper', 'letterpaper', 'a5paper', 'legalpaper',
% 'executivepaper' and 'landscape') and font family ('sans' and 'roman')
\documentclass[11pt,a4paper,roman]{moderncv} 

% moderncv themes
% style options are 'casual' (default), 'classic', 'oldstyle' and 'banking'
\moderncvstyle{classic}

% color options 'blue' (default), 'orange', 'green',
% 'red', 'purple', 'grey' and 'black'
% \moderncvcolor{orange}

\usepackage{happycv}

% to set the default font; use '\sfdefault' for the default sans serif font,
% '\rmdefault' for the default roman one, or any tex font name
% \renewcommand{\familydefault}{\sfdefault}

% uncomment to suppress automatic page numbering for CVs longer than one page
%\nopagenumbers{}

% character encoding
% if you are not using xelatex ou lualatex,
% replace by the encoding you are using
\usepackage{DejaVuSansMono}
\usepackage[utf8]{inputenc}
\usepackage{fontspec,indentfirst}
\usepackage{xunicode}% provides unicode character macros
\usepackage{xltxtra} % provides some fixes/extras
\usepackage{wasysym}

% if you need to use CJK to typeset your resume in Chinese, Japanese or Korean
% \usepackage{CJKutf8}
\usepackage{tikz}
\usepackage{footmisc}

\XeTeXlinebreaklocale "zh"
\XeTeXlinebreakskip = 0pt plus 1pt minus 0.1pt

\newfontfamily\xingkai{"华文行楷"}
\newfontfamily\caiyun{"华文彩云"}
\newfontfamily\kai{"楷体"}
\newfontfamily\fs{"仿宋"}
\newfontfamily\li{"隶书"}
\newfontfamily\xinwei{"华文新魏"}
\newfontfamily\yao{"方正姚体"}
\newfontfamily\hei{"黑体"}
\newfontfamily\song{"新宋体"}
\newfontfamily\mshei{"微软雅黑"}

\setmainfont{"宋体"}

\renewcommand{\baselinestretch}{1.1}
\newenvironment{tightitemize}
{\begin{itemize}\setlength{\parskip}{0pt}}
{\end{itemize}}
% adjust the page margins
\usepackage[scale=0.8]{geometry}

% if you want to change the width of the column with the dates
%\setlength{\hintscolumnwidth}{3cm}
% for the 'classic' style, if you want to force the width allocated to
% your name and avoid line breaks. be careful though, the length is normally
% calculated to avoid any overlap with your personal info;
% use this at your own typographical risks...
%\setlength{\makecvtitlenamewidth}{10cm}

\makeatletter
\tikzset{
    tl@startyear/.append style={
        xshift=(0.5-\tl@startfraction)*\hintscolumnwidth,
        anchor=base
    }
}
\makeatother

% quote, optional, remove / comment the line if not wanted.
% \myquote{MAKE A LITTLE SPACE, MAKE A BETTER PLACE.}{Michael Jackson: <<Heal The World>>}

% to show numerical labels in the bibliography (default is to show no labels);
% only useful if you make citations in your resume
%\makeatletter
%\renewcommand*{\bibliographyitemlabel}{\@biblabel{\arabic{enumiv}}}
%\makeatother

% CONSIDER REPLACING THE ABOVE BY THIS
%\renewcommand*{\bibliographyitemlabel}{[\arabic{enumiv}]}

% bibliography with mutiple entries
%\usepackage{multibib}
%\newcites{book,misc}{{Books},{Others}}
%-------------------------------------------------------------------------------
%                                 content
%-------------------------------------------------------------------------------
\begin{document}
% \begin{CJK*}{UTF8}{gbsn} % to typeset your resume in Chinese using CJK
\makecvtitle

% footnote in my moderncv.
%\footnotetext[1]{http://hbase.apache.org/}
%\footnotetext[2]{https://code.google.com/p/gperftools/}
%\footnotetext[3]{www.neo4j.org}
%\footnotetext[4]{http://zookeeper.apache.org/}


\section{\li{教育经历}}
\tlcventry{2014}{0}{\hei{硕士在读}}{智能技术与系统国家重点实验室}{计算机科学与技术系}{清华大学}{}
\tllabelcventry{2010}{2014}{2010-2014}{\hei{学士}}{计算机科学与技术}{清华大学}{北京}{}

\section{\li{专业技能}}
\cvdoubleitem{操作系统:}{Ubuntu, Android, iOS, Mac OS X}{数据库:}{MySQL, SQLite, Oracle}
\cvdoubleitem{软件编程:}{C/C++, Java, Python}{硬件编程:}{VHDL + Quartus}
\cvdoubleitem{代码工具:}{Vim, Git, Eclipse, X-Code, openGL, openCV}{集成环境:}{Visual Studio, Android Studio, Matlab, Qt,}
\cvdoubleitem{其他工具:}{OPNET, Photoshop,Dreamweaver, HTC Vive+Steam+Unity}{文档编写:}{LaTeX, Word, PPT}
%\cvdoubleitem{个人博客}{\textcolor[rgb]{0.4,0.4,0.4}{\texttt{\httplink{www.cnblogs.com/haippy/}}}}{\textsc{Github} 主页}{\textcolor[rgb]{0.4,0.4,0.4}{\texttt{\httplink{www.github.com/forhappy/}}}}

\section{\li{本硕期间参与项目}}
\tllabelcventry{2011}{0}{2011.10-2011.12}{\hei{用Qt实现迷宫游戏和数独游戏的应用软件}}{全部工作}{编程环境:Qt}{}{
\begin{tightitemize}
	\item 关键技术:~界面编写,界面交互,游戏生成
\end{tightitemize}}
\tllabelcventry{2011}{0}{2011.10-2011.12}{\hei{用Python实现网站信息抓取、过滤与搜索}}{全部工作}{编程语言:Python}{}{
\begin{tightitemize}
	\item 关键技术:~正则表达式,信息匹配及搜索算法
\end{tightitemize}}
\tllabelcventry{2012}{0}{2012.4-2012.6}{\hei{用Java实现新闻更新、阅读、格式转换的软件}}{全部工作}{编程环境:Eclipse}{}{
\begin{tightitemize}
	\item 关键技术:~xml, html, pdf, doc/docx文件格式等间的相互转换
\end{tightitemize}}
\tllabelcventry{2012}{0}{2012.4-2012.6}{\hei{用Java实现的清华校园新闻网的搜索引擎}}{后端开发工作}{编程环境:Eclipse+Tomcat+Heritrix}{}{
\begin{tightitemize}
	\item 关键技术:~网页抓取、关键信息提取、搜索匹配算法
\end{tightitemize}}
\tllabelcventry{2013}{0}{2013.4-2011.6}{\hei{实现足球机器人带球行进射门}}{主要开发人员}{编程语言:C++}{}{
\begin{tightitemize}
	\item 关键技术:~机器人定位及环境障碍分析
\end{tightitemize}}
\tllabelcventry{2014}{0}{2014.3-2014.9}{\hei{基于时延容忍网络的网络协议设计与仿真}}{主要开发人员}{编程环境:OPNET,编程语言:C/C++}{}{
\begin{tightitemize}
	\item 关键技术:~Bundle协议数据格式和协议状态机的设计、实现和仿真
\end{tightitemize}}
\tllabelcventry{2014}{2015}{2014.9-2015.12}{\hei{空间机械臂大时延遥操作系统的建立与研究}}{主要开发人员}{编程环境:Kinect,Visual Studio, Matlab, openGL, openCV,编程语言:C++}{}{
\begin{tightitemize}
	\item 前端界面的编写,工作空间标定,机械臂路径规划
	\item 时延预测,实现键盘和手柄等的控制方式,视觉伺服下自主抓取物体投放到指定地点
\end{tightitemize}}
\tllabelcventry{2016}{0}{2016.7-2016.9}{\hei{冗余机械臂避障路径规划和RRT算法改进}}{全部工作}{编程环境:Matlab,编程语言:Matlab编程语言,论文撰写工具:LaTex}{}{
\begin{tightitemize}
	\item 关键技术:~基于Newton-Raphson算法的路径规划,基于五次多项式的路径平滑,对RRT无目标性的改进算法RRT-GD.
\end{tightitemize}}
\tllabelcventry{2016}{0}{2016.5-至今}{\hei{基于VR技术的虚拟灵巧手系统的建立与研究}}{主要设计及开发人员}{开发环境:HTC Vive+Unity,编程语言:C/C++}{}{
\begin{tightitemize}
	\item 设计目标:~在三维虚拟空间中建立可交互的灵巧手(包含机械臂)模型,实现沉浸式的灵巧手操作体验,来模拟真实灵巧手的行为
\end{tightitemize}}

\section{\li{主要实习经历}}
\tllabelcventry{2012}{0}{2012.7-2012.9}{\hei{北京摩博科技科技有限公司}}{}{}{}{
\begin{tightitemize}
	\item 自学:iOS编程。开发平台:iPad,iPhone,Apple电脑.
	\item 开发软件:每日新闻图片更新显示应用软件,iPad平台,抓取新闻信息,软件界面设计及新闻信息及图片显示
\end{tightitemize}
}

%\footnotetext[5]{有效代码, 下同.}
%\footnotetext[6]{计算机技术与软件专业技术资格(水平)考试}

\section{\li{个人开源项目}}
% \tlcventry{2007--2011}{Degree}{Institution}{City}{\textit{Grade}}{Description}
\tllabelcventry{2016}{0}{2016.9}{\texttt{\textcolor[rgb]{0.55,0,0}{robotRRT}}}{\textsc{Matlab project}}{\textcolor[rgb]{0.4,0.4,0.4}{\texttt{\httplink{www.github.com/gjxswk/robotRRT}}}}{}{
\begin{tightitemize}
	\item 基本介绍:~实现机器人避障路径规划,主要实现方法为RRT及其改进算法,包括bi-RRT及RRT-GD(自己写的改进算法)。逆运动学优化主要采用MLG和Newton-Raphson算法,对于Newton-Raphson算法规划的路径,使用五次多项式进行平滑。
\end{tightitemize}
}

\section{\li{学术论文}}
\cvitem{2014.03}{\kai{《基于时延容忍网络的网络协议设计与仿真》,在OPNET中设计网络协议实现时延容忍网络通信机制,并仿真验证通信机制的性能}}
\cvitem{2016.09}{\kai{《Robot Path Planning Using Newton-Raphson Method Based on RRT-GD》,基于Newton-Raphson方法的RRT避障路径规划改进算法}}

\section{\li{本科及硕士阶段所获主要奖项}}
\subsection{主要奖项}
\cvitem{2010-2014}{\kai{北京兴大助学金}}
\cvitem{2010-2011}{\kai{清华大学优秀学生奖助学金}}
\cvitem{2011.11}{\kai{国家励志奖学金}}

\section{\li{自我评价}}
\cvitem{}{{为人随和, 责任心强, 性格乐观,喜欢跑步, 足球,篮球,旅游等等, 兴趣爱好广泛,喜欢探索新事物\smiley{}}}

% if you are typesetting your resume in Chinese using CJK;
% the \clearpage is required for fancyhdr to work correctly with CJK,
% though it kills the page numbering by making \lastpage undefined
\clearpage
% \end{CJK*}
\end{document}

%% end of file `my-CV-2013.tex'.
